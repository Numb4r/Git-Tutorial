\documentclass[]{article}


\usepackage[backend=biber]{biblatex}
\addbibresource{references.bib}
\title{Tutorial Git}
\author{Yuri Dimitre Dias de Faria}
\date{Fevereiro 2020}

\begin{document}

\maketitle    

\section{O que é Git?}


Git é um sistema de controle de versões distribuido desenvolvido por Linus Torvalds.


Seu objetivo inicial era o versionamento e a geração de um historico de alterações do desenvolvimento do kernel Linux.
Rapidamente o Git foi sendo adotado por outros projeto, sendo eles de desenvolvimento de software ou com propostas gerais.


Git e distribuido sob a licensa GNU General Public License \textcite{GNULicense}. Atualmente sua manutenção e supervisionada por Junio Hamano.

\section{Como instalar o Git(Linux)}


O Git pode vir por padrão em varias distribuições.Para ver se sua distro ja possui ele, basta apenas rodar o comando ``git - -version''.


Para instalar, verifique qual gerenciador de pacotes sua distribuição utiliza.

\subsubsection*{DPKG}

\# apt-get install -y git 

\subsubsection*{RPM}

\# dnf install -y git    

\subsubsection*{PACMAN}

\# pacman -Sy git



\section{Configurações do Git}


Agora que o Git está instalado no sistema, podemos personalizar seu ambiente. As suas customizações se manterão mesmo depois do programa ser atualizado. Alem disso, as configurações poderão ser alteradas a qualquer momento, bastando apenas rodar os comandos novamente. 


O Git vem com uma ferramenta ja pronta para configuração, bastando apeanas rodar o comando \textit{git config} e atribuir valores as variaveis.


Estas variaveis ficam armazenadas em três lugares distintos:


\begin{itemize}
    \item {\textbf{/etc/gitconfig:} } Valido para todos os usuarios do sistema.Para isso, utilize a flag --system do comando git config.
    
    \item {\textbf{\textasciitilde/.gitconfig ou \textasciitilde/.config/git/config:} } Valido apenas para o usuario atual. Para isso, utilize a flag --global do comando git config.
    
    \item {\textbf{config(.git/config) de  cada repositorio:} } Especifico  do repositorio.
\end{itemize}
Você pode ver suas configurações com o comando \textit{git config - -list}.

\subsection*{Sua identidade}
A primeira coisa a ser feita  após o Git ser instalado e a configuração de identidade. Ela é importante ja que todo commit e carimbado com essas informações, que sao imutaveis.

\textbf{\$ git config --global user.name ``Nome'' }

\textbf{\$ git config --global user.email Nome@domain.br}

Note a flag \textit{- -global}, o que significa que essa configuração será valida para o usuario atualmente logado.

\section{Repositorios}

Repositorios são a maneira do Git  trabalhar. Você pode ter mais de um em sua maquina, mas cada projeto possui o seu exclusivo.

Exitem duas formas de adquirir um repositorio, criando ou clonando.

Esse tutorial irá abranger comandos que irão ser executados em emuladores de terminal ou janelas de comando. 
Existem programas, como o GitKraken, que é possivel criar e gerenciar repositorios por meio de interfaces graficas. No entanto, é preciso
um conhecimento previo e bem estabelecido dos comandos Git para sua total usabilidade.

\subsection*{Criando}

Para criar um repositorio, primeiro precisa navegar até a pasta onde está seu projeto. O comando \textit{cd} é usado para mudar o diretorio atual da sua janela de comandos.

cd /users/user/Projects/Tutorial

Assim que estiver na pasta do seu projeto, um repositorio Git pode ser criado pelo comando \textit{git init}. Esse comando irá criar um subdiretorio \textbf{ .git } dentro da pasta do seu projeto.

\subsection*{Clonando}

Para clonar um repositorio existente, navegue ate a pasta que deseja salva-lo e execute \textit{git clone [url]}.

Por exemplo, se quiser clonar o repositorio do youtube-dl no Github,basta apenas executar o comando: 

\textbf{git clone https://github.com/ytdl-org/youtube-dl/}

\section{Alterando um repositorio}

O Git trabalha com status quando se modifica o conteudo do repositorio:

\begin{itemize}
    \item {\textbf{Untracked: }} Arquivos que não foram vinculados ao repositorio ainda.
    \item {\textbf{Unmodified: }} Arquivos do repositorio que não sofreram alterações, seja uma modificação no conteudo ou sua exclusão.
    \item {\textbf{Modified: }} Arquivos que foram modificados, mas suas alterações ainda não foram definidas e enviadas para o repositorio.
    \item {\textbf{Staged: }} Arquivos que foram modificados e serão enviados para o repositorio.
\end{itemize}

Para verificar o status do arquivo, execute o comando \textit{git status}
\subsubsection*{Diff}
É possivel visualizar as alterações de arquivos marcados como ``Modified'' antes de eles passarem para ``Staged''. Basta executar \textit{git diff} que será possivel visualizar as alterações.O que esta de vermelho e a linha possuir um ``-'' significa que foi removido, e o que estiver de verde e a linha possuir um ``+'' significa que foi adicionado.

Você ainda pode visualizar as modificações dos arquivos marcados como ``Staged'' com o ultimo commit adicionando o flag ``- -staged''.

\subsection*{Commit}
Commits são a maneira de gerencia que o Git utiliza. Um commit possui todas as alterações que serão enviadas para o repositorio.
Todos os arquivos com status ``Modified'' serão enviados pelo mudados para ``Staged''.

 Para se adicionar arquivos no commit, utiliza-se o comando \textit{git add [file]}.

 Para se realizar um commit basta usar o comando \textit{git commit}. Você ainda pode incluir uma mensagem com uma informação junto com a flag \textit{-m} e a mensagem a frente.

 
 \textbf{git add *.c}


 \textbf{git commit -m ``Adicionando todos os sources codes''}

 Lembre-se que o commit ainda não foi enviado para o repositorio, ou seja, eles são armazenados localmente.

 Para mais informações, consulte o manual utilizando o comando \textit{man git-commit}
 \subsubsection*{.gitignore}
 É possivel ignorar arquivos ou ate pastas inteiras quando montar seu commit. Para isso, basta criar um arquivo chamado \textit{.gitignore} e colocar o nome dos arquivos ignorados dentro.

 O repositorio \url{https://github.com/github/gitignore} possui alguns .gitignore prontos de varias linguagens.

 
\section{Enviando e recebendo arquivos}

\subsection*{Push}
Para enviar informações para o repositorio remoto, utilizaremos o comando \textit{git push}. Para isso, é preciso que exista algum commit pendente para ser enviado.

É possivel passar alguns argumentos para o comando,como o link do repositorio remoto ou qual a branch será enviado o commit.
Os argumentos mais usados são \textit{origin master}, onde ``origin'' e o repositorio padrão que foi configurado e ``master'' é a branch.

Existe uma infinidade de argumentos e flags para o comando \textit{push}, você pode verificar utilizando o comando \textit{man git-push} ou acessando \url{https://git-scm.com/docs/git-push}

\subsection*{Pull}

Para atualizar os arquivos do repositorio local basta utilizar o comando \textit{git pull}. 
Isso irá mesclar os arquivos da branch atual do repositorio remoto para o repositorio local.Esse comando é uma mescla dos comandos 
\textit{git fetch} e \textit{git merge FETCH\_HEAD}.

\subsection*{Merge}

Merge é um comando para juntar o historico entre branchs.











\printbibliography 
\end{document}